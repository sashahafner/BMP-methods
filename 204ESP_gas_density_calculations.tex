\documentclass[spanish]{article}
\usepackage[version=3]{mhchem}
\usepackage[utf8]{inputenc}
\usepackage{natbib}
\bibliographystyle{abbrvnat}
\usepackage[colorlinks]{hyperref}
\hypersetup{
  citecolor = {blue},
}
\newcommand{\unit}[1]{\ensuremath{\, \mathrm{#1}}}


\title {Metodología para calcular la producción de metano cuando se mide el potencial bioquímico de metano utilizando el método basado en la densidad del biogás\footnote{
  Citación recomendada: 
Hafner, S.D.; Justesen, C.G.; Thorsen, R.; Astals, S.; Holliger, C.; Koch, K.;, Weinrich, S., Calculation of methane production from gas density-based measurements. Standard BMP Methods document 204, version 1.5. Available online: https://www.dbfz.de/en/BMP (accessed on April 19, 2020).
\newline
  En \url{https://www.dbfz.de/en/projects/bmp/methods} encontraréis un documento BibTeX para importarlo al gestor de referencias.
\newline
  Este documento es una traducción al español del documento original en inglés, en caso de duda el documento en inglés prevalece sobre esta traducción.
}
}

\author{Sergi Astals, Sasha D.Hafner, \\ Camilla G. Justesen, Rasmus Thorsen, \\ Christof Holliger, Konrad Koch, \\ and S{\"o}ren Weinrich
\\
\texttt{sastals@ub.edu}\\
}

\date{\today \\
\bigskip
\textit{
  Document number 204.
  Versión en español 1.0. 
  Este documento forma parte de una serie de documentos dedicados a la estandarización de los ensayos de medición del potencial bioquímico de metano.
    \footnote{Para más información y otros documentos visite \url{https://www.dbfz.de/en/BMP}. 
    Para ver el historial de versiones y proponer cambios visite \url{https://github.com/sashahafner/BMP-methods}.}
}
}


\begin{document}
\maketitle

\section{Introducción}
El método para medir el potencial bioquímico de metano utilizando la densidad de biogás (GD-BMP del inglés \textit{gas-density biochemical methane potential}) utiliza la pérdida de masa y el volumen de biogás venteado de uno o más intervalos para determinar la densidad del biogás y, con ello, su composición. Una vez calculada la densidad del biogás, la producción de CH$_4$ se puede calcular con el volumen de biogás venteado o con la pérdida de masa de la botella. Este documento explica detalladamente los cálculos para el método GD-BMP e incluye un ejemplo. El desarrollo y validación del método GD-BMP esta explicado en \citet{justesenDevelopmentValidationLowcost2019}.

\section{Metodología para calcular la producción de metano}
La ecuación (\ref{eq:1}) calcula la densidad del biogás ($\rho_b$, g/mL) utilizando la pérdida de masa ($\Delta m_b$, g) y el volumen de biogás venteado en condiciones estándar (0 $^\circ$C, 101,325 kPa) ($V_b$, mL), corregido por la cantidad de vapor de agua en el biogás ($c\textsubscript{H$_2$O}$, g/mL). 
\begin{equation}
  \label{eq:1}
  \rho_b=\frac{\Delta m_b}{V_b}-c\textsubscript{H$_2$O}
\end{equation}
El volumen de biogás venteado en condiciones estándar ($V_b$) es determinado teniendo en cuenta el volumen de biogás venteado ($V_{meas}$), la presión de vapor de agua, la temperatura y la presión. Ver ecuación (\ref{eq:2}).
\begin{equation}
  \label{eq:2}
  V_{b} = V_{meas} \cdot \frac{(p_{meas} - p\textsubscript{H$_2$O})} {101,325 \unit{kPa}} \cdot \frac {273,15 \unit{K}}{(T_{meas} + 273,15)}
\end{equation}
donde $p_{meas}$ es la presión del biogás cuando se mide el volumen de biogás venteado (aprox. presión atmosférica, kPa), $T_{meas}$ es la temperatura del biogás en $^\circ$C cuando se mide el volumen de biogás, $p \textsubscript{H$_2$O}$ es la presión de vapor (kPa), 273,15 K (0$^\circ$C) es la temperatura estándar y 101,325 kPa es la presión estándar (kPa). El documento 201, dedicado a la determinación del potencial bioquímico de metano utilizando el método volumétrico, proporciona más información sobre este cálculo \citep{BMPdoc201vol}.

La presión de vapor de agua en el biogás ($p\textsubscript{H$_2$O}$, kPa) se calcula, asumiendo que el biogás está saturado, utilizando la ecuación (\ref{eq:3_magnus}) de \citet{alduchovImprovedMagnusForm1996}\footnote{
  Existen otras ecuaciones en la bibliografía, todas ellas proporcionan resultados casi idénticos.
}.

\begin{equation}
  \label{eq:3_magnus}
  p\textsubscript{H$_2$O} = 0,61094 \cdot e^{\frac{17,625 T_{hs}}{243,04 + T_{hs}}}
\end{equation}
donde $T_{hs}$ es la temperatura en el espacio de cabeza de la botella cuando se ventea el biogás ($^\circ$C). 

La concentración de vapor de agua en el biogás venteado ($c\textsubscript{H$_2$O}$) se calcula con la ecuación (\ref{eq:4}) utilizando la masa molecular del agua ($M\textsubscript{H$_2$O}$ = 18.02 g/mol), la presión de vapor del agua en $T_{hs}$ ($p\textsubscript{H$_2$O}$, kPa), la presión del biogás en la botella justo antes de ventear ($p_{hs}$, kPa) y el volumen molar del biogás en condiciones estándar (seco, 101,325 kPa, 0$^\circ$C). El volumen molar del biogás ($v_b$) en condiciones estándar es aproximadamente 22300 mL/mol \citep{hafnerValidationSimpleGravimetric2015} y la presión del biogás justo antes de ventear ($p_{hs}$)se estima en 150 kPa.
\begin{equation}
  \label{eq:4}
  c\textsubscript{H$_2$O}=M\textsubscript{H$_2$O} \cdot \frac{p\textsubscript{H$_2$O}}{p_{hs}-p\textsubscript{H$_2$O}} \cdot \frac{1}{v_b}
\end{equation}
Finalmente, la fracción molar de \ce{CH4} metano ($x_{CH_4}$, adimensional) normalizado para \ce{CH4} y \ce{CO2} ($x_{CH_4}$ + $x_{CO_2}$ = 1) se calcula con la ecuación (\ref{eq:5})  mediante la diferencia normalizada entra la densidad del \ce{CO2} y el biogás. En la ecuación (\ref{eq:5}), $\rho\textsubscript{CH$_4$}$ y $\rho\textsubscript{CO$_2$}$ se refiere a la densidad del gas puro en condiciones estándar. Estos parámetros tienen un valor de 0.0007174 y 0.001977 g mL$^{-1}$ para el \ce{CH4} y el \ce{CO2} respectivamente.
\begin{equation}
  \label{eq:5}
  x\textsubscript{CH$_4$}=\frac{\rho\textsubscript{CO$_2$} - \rho_b}{\rho\textsubscript{CO$_2$}-\rho\textsubscript{CH$_4$}}
\end{equation}
La ecuación \ref{eq:5} calcula el contenido de \ce{CH4} en el biogás, el cual se puede usar para determinar el potencial de metanización con el método gravimétrico o volumétrico \citep{hafnerValidationSimpleGravimetric2015}. La ecuación \ref{eq:5} asume que el biogás solo contiene \ce{CH4} y \ce{CO2}. El gas utilizado para lavar el espacio de cabeza de las botellas al inicio del ensayo no afecta mucho el resultado, pero los cálculos se pueden corregir para evitar esta fuente de error \citep{justesenDevelopmentValidationLowcost2019}.
Finalmente recordar que la aplicación online gratuita OBA \url{ https://biotransformers.shinyapps.io/oba1/ } realiza estos cálculos de forma automática.
\section{Ejemplo} \label{s_example}
En el siguiente ejemplo la producción de \ce{CH4} se calcula para una botella de un ensayo de biometanización. Para todo el ensayo, el volumen de biogás producido en condiciones estándar es 779,2 mL, mientras que la pérdida de masa total es 1,070 g.

Para determinar la densidad del biogás ($\rho_b$) utilizamos la ecuación \ref{eq:1}. La concentración de vapor de agua en el biogás se calcula con la ecuación (\ref{eq:3_magnus}). 
La temperatura del espacio de cabeza (\textit{T}$_{hs}$) se estima en 30$^\circ$C.
\begin{equation*}
  p\textsubscript{H$_2$O} = 0,61094 \cdot e^{\frac{17,625 \cdot 30\unit{^\circ C}}{243,04 + 30 \unit{^\circ C}}} = 4,237 \unit{kPa}
\end{equation*}
A continuación usando la ecuación \ref{eq:4}, se calcula la concentración de vapor de agua en el biogás ($m\textsubscript{H$_2$O}$).
\begin{equation*}
  c\textsubscript{H$_2$O} = 18,016 \unit{g~mol^{-1}} \cdot \frac{4,237 \unit{kPa}}{150 \unit{kPa} - 4,237 \unit{kPa}} \cdot \frac{1 \unit{mol}}{22300 \unit{mL}} = 2,348\cdot10^{-5} \unit{g~mL^{-1}}
\end{equation*}
Con $c\textsubscript{H$_2$O}$, el volumen de biogás venteado en condiciones estándar ($V_{b}$) y la pérdida de masa de la botella ($\Delta m_b$) se calcula la densidad del biogás con la ecuación \ref{eq:1}.
\begin{equation*}
  \rho_b=\frac{1.070 \unit{g}}{779.2 \unit{mL}} - 2.348 \cdot 10^{-5} \unit{g~mL^{-1}} = 1.349 \cdot 10^{-3} \unit{g~mL^{-1}}
\end{equation*}
Finalemente, la fracción molar de CH$_4$ en el biogás ($x\textsubscript{CH$_4$}$, adimensional) se calcula con la ecuación \ref{eq:5}. 

\begin{equation*}
  x\textsubscript{CH$_4$}=\frac{1.977  \cdot 10^{-3} \unit{g~mL^{-1}} - 1.349 \cdot 10^{-3} \unit{g~mL^{-1}}}
  {1.977  \cdot 10^{-3} \unit{g~mL^{-1}} - 7.174 \cdot 10^{-4} \unit{g~mL^{-1}}}
  = 0.498
\end{equation*}

\bibliography{bib}

\end{document}
