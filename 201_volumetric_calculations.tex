\documentclass[]{article}
\usepackage[version=3]{mhchem}
\usepackage{natbib}
\bibliographystyle{abbrvnat}
\usepackage{amsmath}
\usepackage[colorlinks]{hyperref}

\newcommand{\unit}[1]{\ensuremath{\, \mathrm{#1}}}

\title {Calculation of Methane Production from Volumetric Measurements\footnote{
  Recommended citation: 
Hafner, S.D.; L\o jborg, N.; Holliger, C.; Koch, K.;, Weinrich, S., Calculation of Methane Production from Volumetric Measurements. Standard BMP Methods document 201, version 1.7. Available online: https://www.dbfz.de/en/BMP (accessed on April 21, 2020).
\newline
  Or see \url{https://www.dbfz.de/en/BMP} for a BibTeX file that can be imported into citation management software.
}}
\author{Sasha D. Hafner, Nanna L\o jborg, \\ Sergi Astals, Christof Holliger, \\ Konrad Koch, and S{\"o}ren Weinrich\\
\\
\texttt{sasha.hafner@eng.au.dk}
} 

\date{\today \\
\bigskip
\textit{
  Document number 201.
  File version 1.7. 
  This document is from the Standard BMP Methods collection.
    \footnote{For more information and other documents, visit \url{https://www.dbfz.de/en/BMP}. 
    For document version history or to propose changes, visit \url{https://github.com/sashahafner/BMP-methods}.}
}
}


\begin{document}
\maketitle

\section{Introduction}
This document describes calculations for volumetric measurement of biogas.
As with manometric methods, two methods are commonly used to address the problem of biogas dilution with flushing gas, and both are described here: one based on normalized \ce{CH4} concentrations (method 1) and one that explicitly includes estimation of \ce{CH4} in the bottle headspace (method 2).
Expected results from the two methods are effectively identical; differences in cumulative \ce{CH4} production are due only to error in measurement of biogas composition or headspace volume, in addition to small effects of changes in biogas composition over time.

\section{Standardization of measured gas volume}
Both methods use the same approach for standardization of gas volume.
Standard gas volume is calculated by correcting the measured volume $V_{meas}$ for water vapor, temperature, and pressure.

\begin{equation}
  \label{eq:stdvol}
  V_{std} = \frac{V_{meas}(p_{meas} - p\textsubscript{H$_2$O})} {101.325 \unit{kPa}} \cdot \frac {273.15 \unit{K}}{(T_{meas} + 273.15)}
\end{equation}
where $p_{meas}$ is the gas pressure, $T_{meas}$ is the gas temperature at the time of volume measurement in $^\circ$C, $p\textsubscript{H$_2$O}$ the water vapor partial pressure, 273.15 K (0$^\circ$C) is the standard temperature, 101.325 kPa is the standard pressure, and $V_{std}$ is the standardized gas volume.
Eq. (\ref{eq:stdvol}) combines Boyle's and Charles's laws.
Other units can, of course, be used, but standard temperature and pressure must be equivalent (e.g., 1.01325 bar, 1.0 atm) and Eq. (\ref{eq:stdvol}) is based on absolute temperature (note the conversion within the equation by $+~273.15$).

The value of $p\textsubscript{H$_2$O}$ is assumed to be the saturation vapor pressure, and can be calculated using, e.g., the Magnus-form equation given below (Eq. 21 in \citet{alduchovImprovedMagnusForm1996})\footnote{
  Other options exist, and will provide nearly identical values.
}:

\begin{equation}
\label{eq:magnus}
   p\textsubscript{H$_2$O} = 0.61094 e^{(17.625 \cdot T_{meas}/(243.04 + T_{meas}))}
\end{equation}

\section{Calculation of methane production}

With volumetric methods, biogas is typically allowed to accumulate during individual measurement intervals.
Biogas volume and composition are then measured for each interval.
When biogas accumulates in sealed bottles, measured biogas volume is taken as the measured vented (removed) volume.
In other cases, biogas might accumulate in a separate vessel also used to measure volume (e.g., a eudiometer).
Regardless of the method used, standardized biogas volume ($V_{b, std}$) is calculated from the measured biogas volume using Eq. (\ref{eq:stdvol}).

\subsection{Method 1}
In the first method, biogas is assumed to consist of only \ce{CH4} and \ce{CO2} at the time of production (i.e., as produced by the microbial community) and \ce{CH4} production is calculated from vented (removed) biogas only.
With this method, each measurement interval is completely independent of the others; \ce{CH4} production for each interval is determined from $V_{b, std}$ and normalized biogas composition for that interval.
This method is described in \citet{richardsMethodsKineticanalysisMethane1991} (Section 3) and \citet{vdiFermentationOrganicMaterials2016} (Eq. (7)).
Coupled with the assumption that all gas production is biogas, this provides the simplest approach for calculating \ce{CH4} production.

Concentrations of \ce{CH4} and \ce{CO2} are normalized so they sum to 1.0:
\begin{equation}
  x\textsubscript{CH$_4$$, n$} = x\textsubscript{CH$_4$}/(x\textsubscript{CH$_4$} + x\textsubscript{CO$_2$})
\end{equation}
where $x\textsubscript{CH$_4$}$ and $x\textsubscript{CO$_2$}$ are the measured \ce{CH4} and \ce{CO2} concentrations as volume (mole) fraction (possibly including a correction for water vapor--this has no effect here) and $x\textsubscript{CH$_4$$, n$}$ is the normalized \ce{CH4} volume fraction.

Methane production in an individual measurement interval is then calculated from the standardized biogas volume measured in that interval $V_{b,std}$ with:
\begin{equation}
  V\textsubscript{CH$_4$} = x\textsubscript{CH$_4$,$ n$} \cdot V_{b, std}
\end{equation}

Cumulative \ce{CH4} production is taken as the cumulative sum of interval values.

\subsection{Method 2}
Method 2 relies on fewer assumptions, but requires the true concentration of \ce{CH4} (volume fraction) of \ce{CH4} within the bottle headspace, with correction only for water vapor.
Here, \ce{CH4} production in an interval has two components: a vented part that is naturally interval, and a residual headspace part, that is naturally cumulative.
Because determination of \ce{CH4} production in an individual interval depends on both, a subscript $i$ for measurement interval is introduced here.

Vented \ce{CH4} for interval $i$ V\textsubscript{CH$_4, v, i$} is calculated simply as the product of standardized vented (removed) biogas volume and \ce{CH4} mole fraction for that interval:
\begin{equation}
  V\textsubscript{CH$_4$$,v, i$} = x\textsubscript{CH$_4$,$ i$} \cdot V_{b, std, i}
\end{equation}

And headspace \ce{CH4} produced in interval $i$ is given by:
\begin{equation}
  V\textsubscript{CH$_4$$, h, i$} = (x\textsubscript{CH$_4$,$ i$} \cdot V_{h, std, i}) - (x\textsubscript{CH$_4$,$ i-1$} \cdot V_{h, std, i-1})
\end{equation}
where $V_{h, std, i}$ is the post-venting total standardized volume of gas in the bottle headspace (calculated using Eq. (\ref{eq:stdvol})).

Total production in interval $i$ is the sum of the two components.
\begin{equation}
  V\textsubscript{CH$_4$$, i$} = V\textsubscript{CH$_4$$, v, i$} + V\textsubscript{CH$_4$$, h, i$}
\end{equation}

As with method 1, the cumulative sum of these values gives cumulative \ce{CH4} production.
Alternatively (and equivalently), cumulative \ce{CH4} production at the end of interval $i$ can be calculated from the following.

\begin{equation}
  V\textsubscript{CH$_4$$, cum, i$} = \sum_{j = 1}^i {V\textsubscript{CH$_4$$, v, i$}} +  x\textsubscript{CH$_4$,$ i$} \cdot V_{h, std, i}
\end{equation}

\bibliography{bib}

\end{document}
