\documentclass[]{article}
\usepackage[version=3]{mhchem}
\usepackage{natbib}
\bibliographystyle{abbrvnat}
\usepackage{amsmath}
\usepackage[colorlinks]{hyperref}

\title {Calculation of Biochemical Methane Potential (BMP)\footnote{
  Recommended citation: 
Hafner, S.D.; Astals, S.; Holliger, C.; Koch, K.;, Weinrich, S., Calculation of Biochemical Methane Potential (BMP). Standard BMP Methods document 200, version 1.7. Available online: https://www.dbfz.de/en/BMP (accessed on November 13, 2020).
\newline
  Or see \url{https://www.dbfz.de/en/BMP} for a BibTeX file that can be imported into citation management software.
}}
\author{Sasha D. Hafner, Sergi Astals, Christof Holliger, \\ Konrad Koch, and S{\"o}ren Weinrich}

\date{\today \\
\bigskip
\textit{
  Document number 200.
  File version 1.7. 
  French translation by Pierre Buffiere.
  This document is from the Standard BMP Methods collection.
    \footnote{For more information and other documents, visit \url{https://www.dbfz.de/en/BMP}. 
    For document version history or to propose changes, visit \url{https://github.com/sashahafner/BMP-methods}.}
}
}

\begin{document}
\maketitle

\section{Introduction}
Ce document décrit le calcul du potentiel biochimique du méthane (également appelé potentiel biométhane) (BMP) à partir de mesures en laboratoire effectuées dans un test BMP par lots.
Les calculs sont basés sur le volume normalisé \ce{CH4} produit dans des bouteilles avec: 1) inoculum uniquement et 2) avec substrat et inoculum, ainsi que des informations sur la quantité d'inoculum et de matière organique du substrat (généralement sous forme de solides volatils, VS, bien que l'oxygène chimique deman, COD, est également utilisé) ajouté à chaque bouteille.
Pour plus de détails sur le calcul de la production \ce{CH4}, voir les documents spécifiques à la méthode de la collection Standard BMP Methods \footnote{
  Les méthodes disponibles sont: volumétrique (document 201), manométrique (document 202), gravimétrique (document 203) et densité de gaz (document 204), toutes téléchargeables à partir de \url{https://www.dbfz.de/en/BMP}.
}.

\subsection{Sélection d'une durée BMP}
La durée d’évaluation des BMP, c’est-à-dire la durée de l’incubation, doit être au moins aussi longue que la durée nette de 1 \% \footnote{
  Pour une description détaillée, voir le document 100 de \url{https://www.dbfz.de/en/BMP}.
}.
Quoi qu'il en soit, il est important que le temps soit identique pour les flacons d'inoculum seul et d'inoculum + substrat lors du calcul de la BMP.
Cela ne signifie pas qu'il ne peut pas varier entre les substrats à partir du même test BMP - c'est possible.
Différents substrats se dégradent à des vitesses différentes, et par conséquent, certains nécessiteront plus de temps que d'autres pour atteindre le critère.

\subsection{SMP et BMP}
Le terme production de méthane spécifique (SMP) est utilisé pour désigner le rendement \ce{CH4} d'un substrat particulier, dans les mêmes unités que le BMP \footnote{
  Volume normalisé \ce{CH4} (sec, 0 $^\circ$ C, 101.325 kPa, appelé volume `` normal '') par unité de masse de substrat VS ajoutée (par exemple, mL g $^{-1}$, souvent écrit en NmL \textsubscript{CH\textsubscript{4} } g \textsubscript{VS} \textsuperscript{-1}).
}.
Le BMP est en fait simplement du SMP à une seule durée appropriée.
Les courbes SMP, c'est-à-dire les valeurs SMP pour tous les intervalles de mesure, montrent le développement du SMP au fil du temps, reflétant la cinétique d'un test particulier, et sont généralement incluses dans les rapports BMP.
Les calculs décrits ci-dessous sont utilisés à la fois pour SMP et BMP.

\section{Calcul de SMP et BMP}
Ces calculs nécessitent les variables suivantes.
Les unités peuvent différer, mais les unités typiques sont répertoriées ci-dessous.
\begin{itemize}
  \item $V$ \textsubscript{{CH$_4$, $S, i, t$}}, le volume normalisé cumulé de \ce{CH4} produit en flacon $i$ contenant l'inoculum et le substrat au moment $t$ (mL)
  \item $V$ \textsubscript{{CH$_4$, $I, j, t$}}, le volume normalisé cumulé de \ce{CH4} produit en flacon $j$ avec inoculum uniquement (flacon `` blanc '') au temps $t$ (mL)
  \item $m_{I, i}$, la masse d'inoculum (généralement mesurée (masse fraîche)) initialement ajoutée à la bouteille $i$ (g)
  \item $m_{VS, S, i}$, la masse de solides volatils (VS) du substrat initialement ajoutée à la bouteille $i$ (g)
  \item $n$, le nombre de bouteilles répliquées avec inoculum + substrat (généralement 3, le minimum)
  \item $k$, le nombre de bouteilles répliquées avec inoculum uniquement (`` blancs '', généralement 3, le minimum)
\end{itemize}

La productivité du méthane de l'inoculum (mL g $^{-1}$) est calculée séparément pour chaque flacon d'inoculum seul à chaque instant de mesure en utilisant Eq. (\ref{eq:inoc_production}).
\begin{equation}
  \label{eq:inoc_production}
  v\textsubscript{{CH$_4$, $I, j, t$}} = \frac {V\textsubscript{{CH$_4$, $I, j, t$}} } { m_{I, j} }
\end{equation}
Et à partir de ceux-ci, une valeur moyenne est calculée comme
\begin{equation}
  \label{eq:inoc_productivity}
  \bar{v}\textsubscript{{CH$_4$, $I, t$}} = \frac{\sum_{j = 1} ^k v\textsubscript{{CH$_4$, $I, j, t$}}} {k}
\end{equation}
où $k$ = le nombre de flacons d'inoculum uniquement.

Production nette \ce{CH4} à partir de flacons d’inoculum + substrat (ml), c’est-à-dire une estimation de la production de \ce{CH4} dérivée uniquement du substrat \footnote{Ce calcul est basé sur l’hypothèse d’additivité pour la production \ce{CH4}, c’est-à-dire la \ce{CH4} de l'inoculum n'est pas affecté par la présence de substrat. C'est presque certainement incorrect, mais des résultats similaires même en faisant varier le rapport inoculum-substrat et les différences plausibles entre la BMP théorique mesurée et maximale suggèrent que ce n'est pas une grande source d'erreur.} Est calculé comme indiqué dans l'équation. (\ref{eq:net_CH4}).
\begin{equation}
  \label{eq:net_CH4}
  V\textsubscript{CH$_4$, $S, i, t, net$} = V\textsubscript{CH$_4$, $S, i, t$} - \bar{v}\textsubscript{{CH$_4$, $I, t$}} \cdot m_{I, i}
\end{equation}
A noter que les unités de masse d'inoculum $m_{I, i}$ sont totalement dépourvues de pertinence et n'ont aucun effet sur les résultats, pour autant qu'elles soient suffisamment précises.
La masse fraîche (humide) est recommandée, pour plus de simplicité, bien que la masse sèche ou VS puisse être utilisée. \footnote{Toute erreur dans la détermination de la matière sèche de l'inoculum ou de la teneur en VS ici est exactement annulée par la combinaison des équations. (\ref{eq:inoc_production}) et (\ref{eq:net_CH4}), n'a donc aucun effet. Cependant, une détermination précise de la teneur en VS de l'inoculum est nécessaire pour le calcul du rapport inoculum-substrat.}.
L'équation (\ref{eq:net_CH4}) peut être simplifiée si toutes les bouteilles (avec et sans substrat) ont exactement la même masse d'inoculum, mais parce que ce n'est jamais complètement vrai, Eq. (\ref{eq:net_CH4}) doit toujours être utilisé.

Le SMP (mL g $^{-1}$) pour un flacon individuel à une durée donnée est calculé en normalisant la production nette \ce{CH4} par la masse du substrat VS:
\begin{equation}
  \label{eq:yield}
  B_{i, t} = \frac {V\textsubscript{CH$_4$, $S, i, t, net$} }  { m_{VS, S, i}}
\end{equation}
La moyenne de ces valeurs est calculée par substrat:
\begin{equation}
  \label{eq:BMP}
  \bar{B_t} = \frac {\sum_{i = 1} ^n B_{i, t}} {n}
\end{equation}
où $n$ est le nombre de bouteilles répliquées.
Collectivement, toutes les valeurs $\bar{B_t}$ pour un seul substrat ou toutes les valeurs $B_{i, t}$ pour une seule bouteille représentent une courbe SMP.
La BMP pour un substrat est considérée comme $\bar{B_t}$ à une durée appropriée.

\section{Calcul de l'erreur aléatoire}
Le calcul de l'erreur aléatoire dans les estimations de BMP doit inclure au moins deux sources d'erreur: la variation de la production \ce{CH4} parmi les bouteilles de substrat et la variation de la productivité apparente de l'inoculum \ce{CH4} dans les blancs (bouteilles d'inoculum uniquement).
L'incertitude dans la détermination de la quantité de substrat VS ajoutée à chaque bouteille peut parfois être une source d'erreur importante et, bien qu'elle ne soit pas requise, son inclusion est recommandée.
Ici, ces trois sources sont toutes quantifiées en utilisant l'erreur standard et seront appelées $s_{\bar{x},1}$ (rendement du substrat), $s_{\bar{x},2}$ (rendement de l'inoculum) et $s_{\bar{x},3}$ (substrat VS).
Notez que $s_{\bar{x},1}$ et $s_{\bar{x},2}$ peuvent inclure de nombreuses sources d'erreur qui contribuent collectivement à la valeur observée.
Les autres sources d'erreur aléatoire sont supposées faibles: détermination de l'inoculum et de la masse du substrat en particulier.
L'erreur systématique, qui peut être plus importante dans certains cas, n'est pas incluse ici.

Les unités sur les trois erreurs standard sont les unités des estimations finales de BMP, par exemple, mL g $^{-1}$, volume normalisé \ce{CH4} à partir du substrat en mL par g de substrat VS.
Ils peuvent être additionnés pour fournir une estimation totale avec:
\begin{equation}
  \label{eq:se_sum}
  s_{\bar{x},BMP} = \sqrt{\sum_{i=1} ^2 s_{\bar{x},i}^2}
\end{equation}

ou, lorsque les trois sources sont incluses:

\begin{equation}
  \label{eq:se_sum_3}
  s_{\bar{x},BMP} = \sqrt{\sum_{i=1} ^3 s_{\bar{x},i}^2}
\end{equation}
Étant donné $s_{\bar{x},BMP}$, l'écart type $s_{BMP}$ peut être calculé en multipliant par $\sqrt{n}$.

\begin{equation}
  \label{eq:sd}
  s_{BMP} = \sqrt{n} \cdot s_{\bar{x},BMP}
\end{equation}

Cette approche est préférée dans la littérature, bien que l'utilisation de l'erreur standard soit peut-être moins ambiguë \footnote{L'interprétation de cet écart type est ambiguë lorsque le nombre de flacons répliqués n'est pas le même pour les flans et les flacons avec substrat et inoculum (ou lorsque la détermination VS est incluse, lorsqu'elle est basée sur un nombre différent de sous-échantillons).
En revanche, $s_{\bar{x},BMP}$ a une interprétation claire: il s'agit d'une estimation de l'écart-type du BMP $\bar{B_t}$ pour un substrat particulier sur plusieurs expériences hypothétiques sans changement d'erreur systématique.}.
L'écart type relatif (RSD) est généralement indiqué:

\begin{equation}
  \label{eq:rsd}
  RSD = 100\% \cdot \frac{s_{BMP}} {\bar{B}}
\end{equation}

La valeur de $s_{\bar{x},1}$ est calculée à partir des valeurs BMP calculées pour des bouteilles individuelles:
\begin{equation}
  \label{eq:se1_calc}
  s_{\bar{x},1} = \sqrt{ \frac{\sum_{i=1} ^n(B_i - \bar{B})^2} {n(n - 1)} }
\end{equation}

Pour calculer $s_{\bar{x},2}$ l'erreur standard de productivité de l'inoculum \ce{CH4} (par exemple, en ml par g de masse d'inoculum) est d'abord calculée à partir des valeurs individuelles déterminées à partir de chaque blanc individuel.
\begin{equation}
  \label{eq:seI_calc}
  s_{\bar{x},I} = \sqrt{\frac{\sum_{j=1} ^k (v_{CH_4, I, j} - \bar{v}_{CH_4, I})^2} {k(k -1)} }
\end{equation}

Pour chaque bouteille de substrat individuelle, une estimation de $s_{\bar{x},2}$ est ensuite effectuée comme suit:
\begin{equation}
  \label{eq:se2i_calc}
  s_{\bar{x},2,i} = s_{\bar{x},I} \cdot \frac{m_{I, i}} {m_{VS, S, i}}
\end{equation}

L'équation (\ref{eq:se2i_calc}) comprend une hypothèse selon laquelle l'erreur dans la détermination de la masse de l'inoculum est négligeable par rapport à l'erreur dans le rendement de l'inoculum \ce{CH4}, ce qui est raisonnable si la quantité d'inoculum est déterminée par la masse.
Compte tenu de ces valeurs, $s_{\bar{x},2}$ est alors calculé par substrat avec:
\begin{equation}
  \label{eq:se2_calc}
  s_{\bar{x},2} = \sqrt{\frac{\sum_{i=1} ^n s_{\bar{x},2,i}^2} {n}}
\end{equation}

Il peut être plus pratique d'utiliser directement le calcul des valeurs d'écart-type en utilisant les fonctions logicielles disponibles, par exemple \texttt{STDEV()} dans Microsoft Excel ou \texttt{sd()} dans R, au lieu d'utiliser les équations (\ref{eq:se1_calc}) et (\ref{eq:seI_calc}).
Utilisez ensuite les deux expressions suivantes pour obtenir les erreurs standard.

\begin{equation}
  \label{eq:se1_calcb}
  s_{\bar{x},1} = \frac{s_{1} } {\sqrt{n}}
\end{equation}

\begin{equation}
  \label{eq:seI_calcb}
  s_{\bar{x},I} = \frac{s_{I} } {\sqrt{k}}
\end{equation}

La troisième source (et facultative) d'erreur aléatoire ($s_{\bar{x},3}$) est déterminée à partir de l'erreur standard relative dans la détermination de la teneur en VS du substrat $rs_{\bar{x},VS}$ (sans dimension, exprimée en fraction de la teneur en VS).
Cette approche est basée sur l'hypothèse que la principale source d'erreur dans la masse du substrat VS provient de la mesure de la teneur en VS du substrat.
Compte tenu de cette valeur, une estimation de $s_{\bar{x},3}$ peut être faite pour chaque bouteille avec:
\begin{equation}
  \label{eq:se3i_calc}
  s_{\bar{x},3,i} = rs_{\bar{x},VS} \cdot B_{i}
\end{equation}

L'équation (\ref{eq:se3i_calc}) diffère de l'équation. (\ref{eq:se2i_calc}) par inclusion de $B_{i}$ car pour $s_{\bar{x},3}$ l'erreur se propage par division et non par soustraction.
Compte tenu de ces valeurs pour des bouteilles individuelles, $s_{\bar{x},3}$ est calculé à partir de l'Eq. (\ref{eq:se3_calc}).
\begin{equation}
  \label{eq:se3_calc}
  s_{\bar{x},3} = \sqrt{\frac{\sum_{i=1} ^n s_{\bar{x},3,i}^2} {n}}
\end{equation}



\end{document}

