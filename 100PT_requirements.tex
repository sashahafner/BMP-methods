\documentclass[]{article}
\usepackage[version=3]{mhchem}
\usepackage{natbib}
\bibliographystyle{abbrvnat}
\usepackage[colorlinks]{hyperref}
\hypersetup{
  citecolor ={blue},
}
\newcommand{\unit}[1]{\ensuremath{\, \mathrm{#1}}}

 
\title{Requisitos para medição e validação do potencial bioquímico de metano (PBM) \footnote{
  Citação recomendada: 
Holliger, C .; Fruteau de Laclos, H .; Hafner, SD; Koch, K .; Weinrich, S .; Astals, S .; et al. Requisitos para medição e validação do potencial bioquímico do metano (PBM). Documento de métodos PBM padrão 100, versão 1.8. Disponível online: https://www.dbfz.de/en/PBM (acessado em 7 de outubro de 2020).
\new online
  Ou consulte \url{https://www.dbfz.de/en/PBM} para um arquivo BibTeX que pode ser importado para o software de gerenciamento de citações.
}}
\authors{
Christof Holliger, 
H{\'e} l{\`e} ne Fruteau de Laclos,
Sasha D. Hafner, \\
Konrad Koch,
S{\"o} ren Weinrich,
Sergi Astals, \\
Madalena Alves, 
Diana Andrade,
Irini Angelidaki, \\
Lise Appels,
Samet Azman,
Alexandre Bagnoud \\
Urs Baier,
Yadira Bajon Fernandez,
Jan Bartacek, \\
Federico Battista,
David Bolzonella,
Claire Bougrier, \\
Camilla Braguglia,
Pierre Buffi{\`e} re,
Marta Carballa, \\
Arianna Catenacci,
Vasilis Dandikas,
Fabian de Wilde, \\
Sylvanus Ekwe,
Elena Ficara,
Ioannis Fotidis, \\
Jean-Claude Frigon,
Agata Gallipoli,
J{\"o} rn Heerenklage, \\
Pavel Jenicek,
Judith Krautwald,
Ralph Lindeboom, \\
Jing Liu,
Javier Lizasoain,
Rosa Marchetti, \\
Florian Monlau,
Mihaela Nistor,
Hans Oechsner, \\
Jo{\~ a} o V{\'i} tor Oliveira,
Andr{\'e} Pauss,
S{\'e} bastien Pommier, \\
Francisco Raposo,
Thierry Ribeiro,
Christian Schaum, \\
Els Schuman,
Sebastian Schwede,
Mariangela Soldano, \\
Anton Taboada,
Michel Torrijos,
Miriam van Eekert, \\
Jules van Lier, 
Isabella Wierinck \\
} 
 
\date{\today \\
\bigskip
\textit{
  Documento número 100.
  Versão do arquivo 1.8. 
  Este documento faz parte da coleção de métodos PBM padrão.
  Tradução para português (PT_BR) foi realizada por Heleno Quevedo de Lima.
    \footnote{Para mais informações e outros documentos, visite \url{https://www.dbfz.de/en/PBM}. 
    Para o histórico da versão do documento ou para propor alterações, visite \url{https://github.com/sashahafner/PBM-methods}.}
}
}
 
\begin{document}
\maketitle
 
\section{Introdução}
Este documento apresenta os requisitos mínimos para medição e validação do potencial bioquímico do metano (também chamado de potencial do biometano) (PBM) em testes de lote e representa o consenso de mais de 40 pesquisadores de biogás.
A lista de requisitos é baseada em \citet{holligerStandardizationBiomethanePotential2016}, com algumas modificações recentes de critérios de validação, conforme descrito em \citet{hafnerImprovingInterlaboratoryReproducibility2020} e detalhes adicionais sobre a padronização dos cálculos.
Para obter detalhes e muitas recomendações adicionais, consulte estes documentos \citep{holligerStandardizationBiomethanePotential2016, hafnerImprovingInterlaboratoryReproducibility2020}.
 
\section{Requisitos para medição do PBM}
\label{sec: requisitos}
\subsection{Análise de substrato e inóculo}
\label{sec: análise}
  O conteúdo de sólidos voláteis (SV) do inóculo e do substrato é necessário para determinar a relação inóculo/substrato (I/S) e para o cálculo de PBM. Para detalhes sobre as medições dos sólidos totais (TS) e SV podem ser encontrados no documento da US EPA (incluindo um documento gratuito detalhado da US EPA \citep{epaMethod1684Total2001}, bem como \citet{strachDeterminationTotalSolids2016} e \citet{bairdStandardMethodsExamination2017}). 
 
  \begin{enumerate}
    \item Sólidos Totais (ST). Determine a massa do inóculo e todos os substratos após a secagem a 105 $ ^ \circ $ C em triplicata. O ST é necessário apenas para determinação do conteúdo de sólidos voláteis (SV).
    \item Sólidos voláteis (SV). Determine a massa do inóculo e todos os substratos após calcinação da amostra seca a 550 $ ^ \circ $ C em triplicata. O SV é determinado a partir da perda de massa.
  \end{enumerar}
 
\subsection{Configuração e duração do teste}
\label{sec: setup}
\begin{enumerate}
  \item Amostras. 
Todos os ensaios de PBM devem incluir três tipos de amostras: lotes com apenas inóculo (`` brancos ''), com inóculo e celulose microcristalina como controle positivo \nota de rodapé{
      Outros substratos de controle positivo poderiam ser usados ​​no futuro \citep{kochEvaluationCommonSupermarket2020}, mas apenas a celulose teve testes extensivos que foram usados ​​para desenvolver os critérios de validação descritos abaixo na Seção \ref{sec: crit} \citep{hafnerImprovingInterlaboratoryReproducibility2020}.
    }, e com inóculo e substrato.
    \item Replicação. 
Todos os testes devem incluir pelo menos 3 lotes (frascos) para cada condição \nota de rodapé{
      Se um frasco é perdido, por exemplo, quebra, resultando em $ n = 2 $ para qualquer condição, os resultados não podem ser validados.
      Portanto, é prudente incluir 4 réplicas, especialmente para o branco.
      Os valores discrepantes podem ser eliminados se houver um bom motivo para suspeitar que houve um erro na medição (por exemplo, vazamento), mas o número restante de repetições deve ser pelo menos 3.
    }
    O número mínimo de lotes usados ​​em um teste de PBM com um substrato é, portanto, 9 (3 brancos, 3 celulose, 3 substrato).
  \item Duração. 
    Encerre os testes de PBM somente após a produção diária de \ce{CH4} dos lotes individuais, em 3 dias consecutivos, ser $ <$ 1,0 \% do volume líquido acumulado de metano do substrato (lote de substrato menos a média de brancos). 
    Para métodos manuais ou outros onde as medições não são feitas todos os dias, a rescisão pode ocorrer no final do primeiro intervalo de medição de pelo menos 3 dias, onde a taxa de produção cai abaixo do máximo de 1 \% (ou dois ou mais intervalos dessa soma a pelo menos 3 dias, todos com taxas abaixo do máximo de 1 \%).
    Se substratos diferentes forem testados, cada substrato pode ser encerrado quando o mais lento dos 3 lotes replicados atingir o critério de encerramento.
    Os brancos devem continuar enquanto for o lote mais lento (mais recente) com substrato.
    A continuação dos testes além desta duração líquida de 1 \% é aceitável e pode ajudar a garantir que os critérios de validação sejam atendidos (Seção \ref{sec: crit}).
\end{enumerar}
 
\section{Cálculos}
\label{sec: cálculos}
\begin{enumerate}
  \item Processamento de dados.
    O volume \ce{CH4} padronizado (seco, 0 $ ^ \circ $ C, 101,325 kPa) é calculado a partir de dados de laboratório usando métodos padronizados. \Footnote{
      Descrições detalhadas dos cálculos estão disponíveis para os seguintes métodos de medição na coleção de Métodos PBM Padrão (\url{https://www.dbfz.de/en/PBM}): volumétrico (documento 201) \citep{PBMdoc201vol}, manométrico ( documento 202) \citep{PBMdoc202man}, gravimétrico (documento 203) \citep{PBMdoc203grav}, e densidade do gás (documento 204) \citep{PBMdoc204gasdens}.
    }
  \item Unidades PBM.
                PBM é expresso em volume \ce{CH4} padronizado (seco, 0 $ ^ \circ $ C, 101,325 kPa, referido como volume `` normal '') por unidade de massa de matéria orgânica de substrato adicionada (normalmente SV, mas às vezes demanda química de oxigênio (DQO)) (geralmente escrito como NmL \textubscript{CH \textubscript{4}} g \textubscript{SV} \textuperscript{-1}). 
  \item Cálculo do PBM.
    O PBM de todos os substratos (incluindo celulose) é calculado subtraindo a produção de inóculo \ce{CH4} (determinado a partir do branco) da produção bruta (total) de \ce{CH4} do substrato com o inóculo e normalizando pela massa de SV do substrato.
    Quaisquer diferenças na massa do inóculo ou do substrato, entre os lotes, devem ser refletidas nos cálculos.
    Os cálculos devem seguir uma abordagem padronizada \footnote{
      O cálculo do PBM é descrito em detalhes no documento 200 \citep{PBMdoc200PBM}.
    }
  \item Cálculo do desvio padrão do PBM.
    O desvio padrão associado a cada valor PBM médio ($ n \ge 3 $) deve incluir a variabilidade de ambos os brancos e lotes (frascos) com substrato e inóculo, junto com a incerteza na massa de SV do substrato adicionado \nota de rodapé{
      Veja o documento 200 \citep{PBMdoc200PBM}. 
    }
\end{enumerar}
 
\section{Critérios de validação}
\label{sec: crit}
Os resultados do PBM que atendem \textit{todos} os seguintes critérios podem ser descritos como `` validados ''. \Footnote{
Os critérios listados acima estão duplicados no documento 101 \citep{PBMdoc101val}, que foi desenvolvido para simplificar e facilitar a localização desses critérios necessários.
}
Caso contrário, os resultados não são validados e os testes devem ser repetidos.
 
\begin{enumerate}
  \item Todos os componentes necessários do protocolo de medição do PBM listados acima (Seção \ref{sec: requisitos}) são atendidos (incluindo duração) e os cálculos são feitos conforme descrito acima (Seção \ref{sec: cálculos}).
  \item A média do PBM da celulose está entre 340 e 395 NmL \textubscript{CH \textubscript{4}} g \textubscript{SV} \textuperscript{-1}.
  \item O desvio padrão relativo para o PBM de celulose (desvio padrão, incluindo variabilidade nos brancos, nos frascos de substrato e SV do substrato adicionado, dividido pelo PBM médio) não é superior a 6 \%.
\end{enumerar}
 
\bibliography{bib}
 
\end{document}


