\documentclass[]{article}
\usepackage[version=3]{mhchem}
\usepackage{natbib}
\bibliographystyle{abbrvnat}
\usepackage[colorlinks]{hyperref}
\hypersetup{
  citecolor = {blue},
}
\newcommand{\unit}[1]{\ensuremath{\, \mathrm{#1}}}

\title {Requisiti per la misura e validazione del potenziale biochimico metanigeno (BMP)\footnote{
  citare come: 
Holliger, C.; Fruteau de Laclos, H.; Hafner, S.D.; Koch, K.; Weinrich, S.; Astals, S.; et al. Requirements for measurement and validation of biochemical methane potential (BMP). Standard BMP Methods document 100, version 1.9. Available online: https://www.dbfz.de/en/BMP (accessed on February 24, 2021).
\newline
  Oppure visitare \url{https://www.dbfz.de/en/BMP} per un file BibTeX che può essere importato nel software di gestione citazioni.
}}
\author{
Christof Holliger, 
H{\'e}l{\`e}ne Fruteau de Laclos,
Sasha D. Hafner,\\
Konrad Koch,
S{\"o}ren Weinrich,
Sergi Astals, 
Madalena Alves, \\ 
Diana Andrade,
Irini Angelidaki,
Lise Appels, 
Samet Azman, \\
Alexandre Bagnoud
Urs Baier,
Yadira Bajon Fernandez,
Jan Bartacek,\\
Federico Battista,
David Bolzonella,
Claire Bougrier,
Camilla Braguglia, \\
Pierre Buffi{\`e}re,
Marta Carballa,
Arianna Catenacci,
Vasilis Dandikas, \\
Fabian de Wilde,
Sylvanus Ekwe,
Elena Ficara,
Ioannis Fotidis,\\
Jean-Claude Frigon,
Agata Gallipoli,
J{\"o}rn Heerenklage,
Pavel Jenicek,\\
Judith Krautwald,
Ralph Lindeboom,
Jing Liu,
Javier Lizasoain, \\
Rosa Marchetti,
Florian Monlau,
Mihaela Nistor,
Hans Oechsner,\\
Jo{\~a}o V{\'i}tor Oliveira,
Andr{\'e} Pauss,
S{\'e}bastien Pommier,
Francisco Raposo, \\
Thierry Ribeiro,
Christian Schaum,
Els Schuman,
Sebastian Schwede, \\
Mariangela Soldano,
Anton Taboada,
Michel Torrijos,
Miriam van Eekert,\\
Jules van Lier, 
Isabella Wierinck
} 

\date{\today \\
\bigskip
\textit{
  Documento \#100.
  Questa è una traduzione spagnola di Camilla Braguglia, Elena Ficara, e Mariangela Soldano del documento inglese (versione 1.9). In caso di dubbio, il documento in inglese prevale su questa traduzione.  
  Questo documento è tratto dalla raccolta dei metodi di BMP standard.
    \footnote{per maggiori informazioni e per altri documenti, vistare la pagina \url{https://www.dbfz.de/en/BMP}. 
    Per la cronologia delle versioni e/o proporre delle modifiche visitare la pagina \url{https://github.com/sashahafner/BMP-methods}.}
}
}

\begin{document}
\maketitle

\section{Introduzione}
Questo documento riporta i requisiti minimi richiesti per la misura e la validazione del potenziale biochimico metanigeno (BMP) valutato mediante prova in batch.
Rappresenta il consenso di oltre 50 ricercatori sul biogas.
L'elenco dei requisiti è lo stesso del commento ad accesso aperto di \citet{holligerStandardizationBiomethanePotential2021}.
Per i dettagli sullo sviluppo di questi requisiti vedere i documenti ad accesso aperto \citet{holligerStandardizationBiomethanePotential2016} e \citet{hafnerImprovingInterlaboratoryReproducibility2020}. 

\section{Requisiti per la misura del BMP}
\label{sec:requirements}
\subsection{Analisi del substrato e dell’inoculo}
\label{sec:analysis}
La misura della concentrazione dei solidi volatili (SV) dell’inoculo e del substrato è necessaria sia per poter determinare le rispettive quantità da utilizzare per un fissato rapporto substrato/inoculo, sia per poter calcolare il valore del (BMP).
   Per ulteriori dettagli sulla misura di solidi totali (ST) e SV consultare \citet{epaMethod1684Total2001} (accesso aperto), o altre fonti (\citet{strachDeterminationTotalSolids2020} (accesso aperto) o \citet{bairdStandardMethodsExamination2017}. 
  \begin{enumerate}
    \item Solidi totali (ST). Analisi in triplicato sia per l’inoculo che per il substrato, condotta essiccando i campioni in stufa a 105$^\circ$C. La misura degli ST è necessaria per poter determinare il contenuto dei solidi volatili (SV).
    \item Solidi volatili (SV). Analisi, in triplicato, sia per l’inoculo che per il substrato, condotta per calcinazione a 550$^\circ$C dei campioni essiccati in precedenza. Il contenuto dei solidi volatili viene determinato dalla perdita di peso.

  \end{enumerate}

\subsection{Setup del test e durata}
\label{sec:setup}
\begin{enumerate}
  \item Campioni. 
    Ogni test BMP deve comprendere tre condizioni: bianco (solo inoculo), controllo positivo (cellulosa microcristallina\footnote{
      In futuro si potranno usare anche altri controlli positivi \citep{kochEvaluationCommonSupermarket2020} ma al momento solo la cellulosa è stata sottoposta ad un ampio set di prove che ha permesso poi di sviluppare i criteri di validazione descritti i nella sezione \ref{sec:crit} \citep{hafnerImprovingInterlaboratoryReproducibility2020}.
    }, ed inoculo), e substrato di prova (substrato ed inoculo).
  \item Repliche. 
    Ogni test BMP deve includere almeno 3 repliche per ogni condizione.\footnote{
      Se una bottiglia viene persa, ad esempio, per rottura, risultando in $n=2$ per qualsiasi condizione, i risultati non possono essere convalidati.
      Pertanto è prudente includere 4 repliche, soprattutto per gli spazi vuoti.
      I valori anomali possono essere eliminati se c'è una buona ragione per sospettare che ci sia stato un errore nella misurazione (ad esempio, perdita) ma il numero rimanente di repliche deve essere almeno 3. 
    }
    Il numero minimo di reattori batch usati per il test BMP di un substrato è pertanto 9 (3 bianchi, 3 controlli positivi, e 3 substrati di prova).
  \item Durata. 
    Il test BMP va interrotto solo quando, per ogni singolo reattore, la produzione giornaliera di metano accumulato, per tre giorni consecutivi, è minore dell’1\% del volume netto (volume prodotto dal substrato meno il volume medio prodotto dai bianchi) (criterio di fine prova). 
    Per misurazioni manuali o per misurazioni che non vengono eseguite ogni giorno, il test può essere interrotto alla fine del primo intervallo di misure di almeno 3 giorni durante il quale il tasso di produzione è minore dell’1\% (oppure 2 o più intervalli di tempo che combinati insieme arrivano almeno a 3 giorni, tutti con tasso minore dell’1\%). 
    Se vengono testati diversi substrati contemporaneamente, ogni substrato può essere interrotto quando la più lenta delle 3 repliche soddisfa il criterio di fine prova. 
    I bianchi devono continuare fino a quando il più lento (o l’ultimo) reattore batch con il substrato non soddisfa i criteri di fine prova. 
    Proseguire la prova oltre la durata così determinata è comunque accettabile e può aiutare a garantire che i criteri di validazione siano raggiunti (vedi sezione \ref{sec:crit}).
\end{enumerate}

\section{Calcoli}
\label{sec:calculations}
\begin{enumerate}
  \item Elaborazione dei dati.
    Il volume normale di \ce{CH4} (gas secco, 0$^\circ$C, 101.325 kPa) viene calcolato dai dati di laboratorio usando i metodi standardizzati.\footnote{
      Una descrizione dettagliata dei calcoli è disponibile per ogni metodo di misura del metano nella raccolta di metodi standard dei BMP alla pagina (\url{https://www.dbfz.de/en/BMP}): volumetrico (documento 201) \citep{BMPdoc201vol}, manometrico (documento 202) \citep{BMPdoc202man}, gravimetrico (documento 203) \citep{BMPdoc203grav}, e tramite densità del gas (documento 204) \citep{BMPdoc204gasdens}.
    }
  \item Unità di misura del BMP.
    Il BMP è espresso pertanto in volume normale di \ce{CH4} per unità di massa di sostanza organica del substrato aggiunta (solitamente quantificata in termini di SV ma a volte anche in termini di COD, domanda chimica di ossigeno); spesso definito come NmL\textsubscript{CH\textsubscript{4}} g\textsubscript{SV}\textsuperscript{-1}). 
  \item Calcolo del BMP. 
    Il BMP del substrato (cosi come quello della cellulosa) viene calcolato sottraendo dalla produzione totale del reattore batch con substrato la produzione di metano dell’inoculo (determinata dai bianchi), e normalizzando rispetto alla quantità di SV del substrato. 
    I calcoli devono comunque tenere conto delle eventuali differenze nella quantità di inoculo e di substrato tra i diversi reattori batch. I calcoli devono seguire un approccio standardizzato.\footnote{
      I calcoli per determinare il BMP sono descritti dettagliatamente nel documento 200 \citep{BMPdoc200BMP}.
    }
  \item Calcolo della deviazione standard del valore di BMP. 
    La deviazione standard, associata ad ogni valore medio ($n \ge 3$) di BMP, deve comprendere la variabilità sia dei bianchi che dei reattori con substrato, insieme all’incertezza della misura dei SV aggiunti.\footnote{
      Vedi documento 200 \citep{BMPdoc200BMP}. 
    }
\end{enumerate}

\section{Criteri di validazione}
\label{sec:crit}
I risultati dei BMP che soddisfano tutti i seguenti criteri possono essere considerati ``validati''.\footnote{
  I criteri elencati sopra sono duplicati nel documento 101 \citep{BMPdoc101val}, creato appositamente per reperire i criteri richiesti più facilmente.
}
Altrimenti i risultati non sono validati, e le prove vanno ripetute.

\begin{enumerate}
  \item Tutti i requisiti richiesti dal protocollo di misura elencati precedentemente (Sezione \ref{sec:requirements}) sono soddisfatti (incluso la durata) ed i calcoli sono stati eseguiti come descritto nella Sezione \ref{sec:calculations}.
  \item Il BMP medio della cellulosa risulta compreso tra 340 and 395 NmL\textsubscript{CH\textsubscript{4}} g\textsubscript{SV}\textsuperscript{-1}.
  \item Il coefficiente di variazione del BMP della cellulosa (inteso come il rapporto tra la deviazione standard, valutata come descritto al punto 3.4, ed il valore medio del BMP) non deve essere maggiore del 6\%.
\end{enumerate}

\bibliography{bib}

\end{document}
