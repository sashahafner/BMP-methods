\documentclass[]{article}
\usepackage[version=3]{mhchem}
\usepackage{hyperref}

\newcommand{\unit}[1]{\ensuremath{\, \mathrm{#1}}}

\title {General protocol for measurement of biochemical methane potential (BMP) (document no. 100)}
\author{Christof Holliger, Sasha D. Hafner, . . .  } 

\begin{document}
\maketitle

\section{BMP Methods collection}
Document number 100.
File version 1.0. 
This file is from the DBFZ BMP Methods collection.
For more information, visit the site at \url{https://www.dbfz.de/en/BMP}.

\section{Description}
This document describes requirements for measurement of biochemical methane potential (also called biomethane potential) (BMP) in batch tests.
For details on the development of this protocol, see Holliger et al. \cite{iis2016} and Hafner et al. \cite{iis2020}.

\section{BMP test overview}
To measure the BMP of a substrate, it is mixed with an anaerobic inoculum in bottles, which are then sealed and incubated. 
Methane \ce{CH4} production is measured over time.
Production of \ce{CH4} from the inoculum is estimated by making similar measurements on bottles with only inoculum (``blanks'').
For more details on the basic approach, see Owen et al. \cite{owen1979}.

\section{Protocol}
\subsection{Inoculum}
\begin{enumerate}
  \item Origin. Must be taken from mesophilic digester (35-40$^\circ$C). Highly diverse methanogenic microbial community has to be present.
  \item Treatment. Generally avoid. Sieving or other pre-treatment (grinding, dilution) acceptable if needed.
  \item Analysis.
    \begin{enumerate}
      \item Total solids (TS) by drying for at least 24 hours at 105$^\circ$C in triplicate
      \item Volatile solids (VS) by combusting at 550$^\circ$C for at least 2 hours in triplicate.
    \end{enumerate}
  \item Quality check before use.\\
    Required:
    \begin{enumerate}
      \item pH between 7.0 and 8.5.
      \item Alkalinity $\ge$ 3 g L$^{-1}$ as \ce{CaCO3}.
    \end{enumerate}

    Optional:
    \begin{enumerate}
      \item Total ammonical nitrogen (TAN) $<$ 2.5 g L$^{-1}$ as N
      \item Total volatile fatty acids (VFAs) $<$ 1.0 g L$^{-1}$ as acetic acid
    \end{enumerate}
  \item Storage. Storage time between collection and setting up BMP tests $\le$ 5 days at ambient (20-25$^\circ$C) or mesophilic test temperature.
  \item Methane production of the blank should be less than 40\% of the methane production of cellulose. 
\end{enumerate}

\subsection{Test setup}
\begin{enumerate}
  \item Samples and replication. 
    All tests must include at least 3 batches (bottles) each with: inoculum only (``blanks''), a positive control substrate (microcrystaline cellulose recommended), and each substrate. 
    All bottles must contain inoculum.
  \item Substrate quantity and bottle size. At least 1.0 g of substrate VS must be added to each bottle. This affects bottle size, and care should be taken to avoid high pressure in manual methods (due to leaks or bottle breakage). Recommended maximum headspace pressure is 2 bar (gauge). Between 5-10 g substrate VS per L headspace volume is recommended as long as daily sampling is possible at the start. 
  \item Inoculum-to-substrate ratio (ISR). On a VS basis, ISR should generally be 2, but may be as low as 1 for slowly degradable substrates, and as high as 4 for easily degradable substrates.
  \item Amendments. 
    \begin{enumerate}
      \item Trace element and vitamin amendment is required.\footnote{
          Trace element solution (concentration in g L$^{-1}$): 2 \ce{FeCl2\cdot4H2O}, 0.05 \ce{H3BO3}, 0.05 \ce{ZnCl2}, 0.038 \ce{CuCl2\cdot2H2O}, 0.05 \ce{MnCl2\cdot4H2O}, 
          0.05 \ce{(NH4)6Mo7O24\cdot4H2O}, 0.05 \ce{AlCl3}, 0.05 \ce{CoCl2\cdot6H2O}, 0.092 \ce{NiCl2\cdot6H2O}, 0.5 ethylenediaminetetraacetate, 1 mL concentrated HCl, 
          0.1 \ce{Na2SeO3\cdot5H2O}.
          \newline
          Vitamin mixture (concentration in mg L$^{-1}$): 2 Biotin, 2 folic acid, 10 pyridoxine acid, 5 riboflavin, 5 thiamine hydrochloride, 0.1 cyanocobalamine, 
          5 nicotinic acid, 5 P-aminobenzoic acid, 5 lipoic acid, X???? DL-pantothenic acid.
          \newline
          Add between 1 and 5 mL of each solution per 1 L of final slurry volume (typically 1 mL each per bottle).
        }.
      \item If alkalinity is too low, add \ce{NaHCO3} to meet requirement.
    \end{enumerate}
  \item Headspace flushing. Flush headspace prior to incubation to remove \ce{O2}. 
    Use a mixture of \ce{N2} and \ce{CO2} (20-40\% \ce{CO2}) or 100\% \ce{N2}. Do not flush liquid phase with pure \ce{N2}. 
    It is recommended to measure gas flow rate and to replace at least 3 headspace volumes.
  \item Incubation.
    \begin{enumerate}
      \item Temperature controlled environment at mesophilic temperature (35-40$^\circ$C) with $\le$ 2$^\circ$C variation during incubation. 
        The temperature should match the temperature of the digester that was the source of inoculum.
      \item Mixing is compulsory, if manually at least once a day.
    \end{enumerate}
  \item Methane production measurement.
    \begin{enumerate}
      \item No restrictions on which system to use
      \item If gas composition has to be analyzed, it has to be analyzed at each measuring point and for every single batch (bottle).
      \item At each measuring point, ambient pressure and temperature has to be measured and recorded for use in gas volume standardization
    \end{enumerate}
  \item Duration. 
    Terminate BMP tests only after daily \ce{CH4} production during 3 consecutive days is $<$ 0.5\% of the net accumulated volume of methane from the substrate (substrate minus average of blanks). 
    This is refered to as the ``1\% net duration''.
    If different substrates are tested, each substrate can be terminated when the slowest of the 3 batches (bottles) has reached the termination criterion.
    Blanks must be continued as long as the slowest (latest) batch (bottle) with substrate.
    Continuing tests beyond the 1\% net duration is acceptable.
\end{enumerate}

\subsection{Calculations}
Details on calculations can be found at \url{https://www.dbfz.de/en/BMP} both for specific methods and for BMP calculation from standardized gas volumes.
\begin{enumerate}
  \item Data processing.
    Standardized \ce{CH4} volume (dry, 0$^\circ$C, 101.325 kPa) must be calculated from raw laboratory data (e.g., measured volume, pressure, mass, or concentration) using accepted methods, if available. 
    Available methods can be found at \url{https://www.dbfz.de/en/BMP}.
    Checking calculations by comparison to available software is recommended.
  \item BMP units.
    BMP should be expressed in standardized \ce{CH4} volume per g of substrate VS added. 
  \item Calculation of BMP.
    BMP of all substrates (including positive control) must be calculated by subtracting inoculum \ce{CH4} production (determined from blanks) from gross (total) \ce{CH4} production from substrate with inoculum, and normalizing by substrate VS mass.
    Calculations should follow the accepted approach from \url{https://www.dbfz.de/en/BMP}.
    Checking calculations by comparison to available software is recommended.
  \item Calculation of BMP standard deviation.
    The standard deviation associated with each mean ($n = 3$) BMP value must include variability from both blanks and batches (bottles) with substrate and inoculum, following the details given in the standard approach \url{https://www.dbfz.de/en/BMP}.
    Inclusion of variability from substrate VS determination is recommended.
\end{enumerate}

\subsection{Validation criteria}
BMP results that meet \textit{all} the following criteria should be considered ``validated'' by the standards of \cite{iis2020}.
Otherwise, results are not validated, and tests should be repeated if possible, and otherwise, the lack of validation should be made clear in any reporting of the results.

\begin{enumerate}
  \item All required components of the BMP measurement protocol listed above are met.
  \item Mean cellulose BMP is between 340 and 395 NmL g$^{-1}$ (standardized \ce{CH4} volume (dry, 0$^\circ$C, 101.325 kPa) per g substrate VS).
  \item Cellulose relative standard deviation (including variability in both blanks and substrate bottles) is no more than 6\%.
\end{enumerate}

\begin{thebibliography}{1}

\bibitem{iis2016}
Holliger, C., Alves, M., Andrade, D., Angelidaki, I., Astals, S., Baier, U., Bougrier, C., Buffi{\`e}re, P., Carballa, M., de Wilde, V., Ebertseder, F., Fern{\'a}ndez, B., Ficara, E., Fotidis, I., Frigon, J.-C., Fruteau de Laclos, H., S. M. Ghasimi, D., Hack, G., Hartel, M., Heerenklage, J., Sarvari Horvath, I., Jenicek, P., Koch, K., Krautwald, J., Lizasoain, J., Liu, J., Mosberger, L., Nistor, M., Oechsner, H., Oliveira, J. V., Paterson, M., Pauss, A., Pommier, S., Porqueddu, I., Raposo, F., Ribeiro, T., R{\"u}sch Pfund, F., Str{\"o}mberg, S., Torrijos, M., van Eekert, M., van Lier, J., Wedwitschka, H., Wierinck, I.
\newblock{2016},
\newblock{Towards a standardization of biomethane potential tests}
\newblock{Water Science and Technology} 74: 2515-2522

\bibitem{iis2020}
  Hafner, S.D., Fruteau de Laclos, H., Koch, K., Holliger, C.
\newblock{2020},
\newblock{Improving inter-laboratory reproducibility in measurement of biochemical methane potential (BMP)}
\newblock{Water}

\bibitem{bmpprotocol}
Holliger, C., . . .
    \newblock{2020},
    \newblock{General protocol for measurement of biochemical methane potential (BMP)},
    \newblock{\url{https://github.com/sashahafner/BMP-methods}}


\bibitem{owen1979}
Owen, W. F., Stuckey, D. C., Healy Jr, J. B., Young, L. Y., McCarty, P. L.
\newblock{1979},
\newblock{Bioassay for monitoring biochemical methane potential and anaerobic toxicity}
\newblock{Water Research} 13: 485-492.


\end{thebibliography}


\end{document}
