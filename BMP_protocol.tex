%\documentclass[twocolumn]{article}
\documentclass[]{article}
\usepackage[version=3]{mhchem}
\usepackage{sectsty}
%\usepackage{amsmath}
%\usepackage{amssymb}
%\usepackage{enumitem}
\usepackage[flushleft]{threeparttablex} % For table notes
\usepackage{rotating} 
\usepackage{longtable}
\usepackage{hyperref}
% For decimal alignment in tables
\usepackage{dcolumn}

% For dcolumn
\newcolumntype{.}{D{.}{.}{3}}

\sectionfont{\large}

\newcommand{\unit}[1]{\ensuremath{\, \mathrm{#1}}}

\title {General protocol for measurement of biochemical methane potential (BMP)}
\author{Christof Holliger, . . .  } 

\begin{document}
\maketitle

\section{BMP-methods}
File version 0.1. 
This file is from the GitHub repository BMP-methods.
For more information, visit BMP-methods at \url{https://github.com/sashahafner/BMP-methods}.

\section{Description}
This document describes general requirements for measurement of biochemical methane potential (also called biomethane potential) in batch tests.

\section{BMP test overview}
To measure the BMP of a substrate, it is mixed with an anaerobic inoculum, added to sealed bottles, incubated, and methane \ce{CH4} production is measured over time.
Production of \ce{CH4} from the inoculum is estimated by making similar measurements on bottles with only inoculum (``blanks'').

\section{Protocol}
\subsection{Inoculum}
\begin{enumerate}
    %[$\square$]
  \item Origin. Must be taken from mesophilic digester (35-40$^\circ$C). Highly diverse methanogenic microbial community has to be present.
  \item Treatment. Generally avoid. Sieving or other pre-treatment (grinding, dilution) acceptable if needed.
  \item Analysis.
    \begin{enumerate}
      \item Total solids (TS) by drying for at least 24 hours at 105$^\circ$C in triplicate
      \item Volatile solids (VS) by combusting at 550$^\circ$C for at least 2 hours in triplicate.
    \end{enumerate}
  \item Quality check before use.\\
    Required:
    \begin{enumerate}
      \item pH between 7.0 and 8.5.
      \item Alkalinity $\ge$ 3 g L$^{-1}$ as \ce{CaCO3}.
    \end{enumerate}

    Optional:
    \begin{enumerate}
      \item Total ammonical nitrogen (TAN) $<$ 2.5 g L$^{-1}$ as N
      \item Total volatile fatty acids (VFAs) $<$ 1.0 g L$^{-1}$ as acetic acid
    \end{enumerate}
  \item Storage. Storage time between collection and setting up BMP tests $\le$ 5 days at ambient (20-25$^\circ$C) or mesophilic test temperature.
  \item Methane production of the blank should be less than 40\% of the methane production of cellulose. 
\end{enumerate}

\subsection{Test setup}
\begin{enumerate}
  \item Replication. All conditions in triplicate: at least 3 batches/bottles each for blanks (inoculum-only), cellulose (positive control), and each substrate. 
  \item Bottle size. Large enough to accept $\ge$ 1.0 g of substrate VS.
  \item Inoculum-to-substrate ratio (ISR). On a VS basis, generally 2, as low as 1 for slowly degradable substrates, and as high as 4 for easily degradable substrates.
  \item Amendments. 
    \begin{enumerate}
      \item Trace element and vitamin amendment\footnote{
          Trace element solution (concentration in g L$^{-1}$): 2 \ce{FeCl2\cdot4H2O}, 0.05 \ce{H3BO3}, 0.05 \ce{ZnCl2}, 0.038 \ce{CuCl2\cdot2H2O}, 0.05 \ce{MnCl2\cdot4H2O}, 
          0.05 \ce{(NH4)6Mo7O24\cdot4H2O}, 0.05 \ce{AlCl3}, 0.05 \ce{CoCl2\cdot6H2O}, 0.092 \ce{NiCl2\cdot6H2O}, 0.5 ethylenediaminetetraacetate, 1 mL concentrated HCl, 
          0.1 \ce{Na2SeO3\cdot5H2O}.
          \newline
          Vitamin mixture (concentration in mg L$^{-1}$): 2 Biotin, 2 folic acid, 10 pyridoxine acid, 5 riboflavin, 5 thiamine hydrochloride, 0.1 cyanocobalamine, 
          5 nicotinic acid, 5 P-aminobenzoic acid, 5 lipoic acid, X???? DL-pantothenic acid.
          \newline
          Add between 1 and 5 mL of each solution per 1 L of final slurry volume (typically 1 mL each per bottle).
        }.
      \item If alkalinity is too low, add \ce{NaHCO3} to meet requirement.
    \end{enumerate}
  \item Headspace flushing. Flush headspace prior to incubation to remove \ce{O2}. 
    Use a mixture of \ce{N2} and \ce{CO2} (20-40\% \ce{CO2}) or 100\% \ce{N2}. Do not flush liquid phase with pure \ce{N2}. 
    It is recommended to measure gas flow rate and to replace at least 3 headspace volumes.
  \item Incubation.
    \begin{enumerate}
      \item Temperature controlled environment at mesophilic temperature (35-40$^\circ$C) with $\le$ 2$^\circ$C variation during incubation. 
        The temperature should match the temperature of the digester that was the source of inoculum.
      \item Mixing is compulsory, if manually at least once a day.
    \end{enumerate}
  \item Methane production measurement.
    \begin{enumerate}
      \item No restrictions on which system to use
      \item If gas composition has to be analyzed, it has to be analyzed at each measuring point and for every single batch (bottle).
      \item At each measuring point, ambient pressure and temperature has to be measured and recorded for use in gas volume standardization
    \end{enumerate}
  \item Duration. Terminate test as late as possible but not before daily \ce{CH4} production during 3 consecutive days is $<$ 0.5\% of the net accumulated volume of methane from the substrate (substrate minus average of blanks). 
    If different substrates are tested, each substrate can be terminated when the slowest batch (bottle) has reached the termination criterion.
    Blanks must be continued as long as any bottles with substrate.
\end{enumerate}

\end{document}
