\documentclass[]{article}
\usepackage[version=3]{mhchem}
\usepackage{natbib}
\usepackage{graphicx}
\usepackage{float}
\bibliographystyle{abbrvnat}
\usepackage[colorlinks]{hyperref}
\hypersetup{
  citecolor = {blue},
}
\newcommand{\unit}[1]{\ensuremath{\, \mathrm{#1}}}

\title {Gravimetric Measurement of Biochemical Methane Potential (BMP)\footnote{
  Recommended citation: 
Hafner, S.D.; Astals, S.; Holliger, C.; Justesen, C.; Koch, K.; Mortensen, J.R.; Richards, B.K. 2023 Gravimetric Measurement of Biochemical Methane Potential. Standard BMP Methods document 303, version 1.0. Available online: https://www.dbfz.de/en/BMP (accessed on \textit{date}).
\newline
  Or see \url{https://www.dbfz.de/en/BMP} for a BibTeX file that can be imported into citation management software.
}
}
\author{Sasha D. Hafner, Sergi Astals, Christof Holliger,\\ Camilla Justesen, Konrad Koch, Jacob R. Mortensen,\\ S\"oren Weinrich, Brian K. Richards} 

\date{\today \\
\bigskip
\textit{
  Document number 303.
  File version 1.0. 
  This document is from the Standard BMP Methods collection.
    \footnote{For more information and other documents, visit \url{https://www.dbfz.de/en/BMP}. 
    For document version history or to propose changes, visit \url{https://github.com/sashahafner/BMP-methods}.}
}
}

\begin{document}
\maketitle

\section{Introduction}
By measuring BMP bottle mass loss due to biogas venting, along with biogas composition, methane (\ce{CH4}) production and biochemical methane potential (BMP) can be determined.
This document describes the laboratory measurements needed for applying this ``gravimetric method''.
Development and validation of the method is described in \citet{hafnerValidationSimpleGravimetric2015}.
Comparisons with other methods can be found in \citet{hafnerSystematicErrorManometric2019}, \citet{amodeoHowDifferentAre2020a}, and \citet{hafnerImprovingInterlaboratoryReproducibility2020}.
For information on gravimetric calculations, see document 203 from the Standard BMP Methods website \citep{BMPdoc203grav}. 
The gravimetric method is very similar to the gravimetric method described in Document 304 \citep{BMPdoc204gasdens}.
Much of the text is identical between the two documents.

\section{Equipment and supplies}
\label{sec:equipment}
For application of the gravimetric method the following laboratory equipment and supplies are required:
\begin{itemize}
    \item Electronic scale
    \item Gas sampling equipment (e.g., syringe and needles)
    \item Gas chromatograph or equivalent system for gas analysis
    \item Typical BMP bottles and septa
    \item Incubator
\end{itemize}

The required accuracy of the scale will depend on the quantity of biogas produced. 
Generally, stated accuracy\footnote{
  Manufacturers often report accuracy as ``linearity''. 
  Note that accuracy is not the same as ``readability'', which is the smallest value that can be read. 
} of the scale should be at least 30 mg for every g of substrate volatile solids (VS) used.\footnote{
  For example, with 2 g of substrate VS added to each bottle, scale accuracy stated by the manufacturer must be 60 mg or better (e.g., 50 mg would be sufficient).
}
As described below, the accuracy and stability of the scale is checked as part of the protocol.

An incubator or temperature-controlled room is needed to keep bottles at the desired test temperature, e.g., 37 $^\circ$C.
A water bath is not recommended because water on a bottle surface will affect its weight.
Incubation of the bottles in heated air is preferable.
Ideally, venting and weighing should be done inside a temperature-controlled room, so bottles are always at the incubation temperature and the headspace temperature -- needed for calculations -- is known.
However, the effect of headspace temperature error on accuracy is very small, so this is not required.
A mechanical mixer is not needed; careful manual mixing at the time of sampling has been found to be sufficient (see Section \ref{sec:incsam}).\footnote{But gentle mechanical mixing could work, as long as the inside of the septum is kept clean.}

The gravimetric method requires separate analysis of biogas composition.
Generally, this will be done by gas chromatography.
Details on operation of a gas chromatograph are not presented here.

\section{Setup}
\label{sec:setup}
During setup, inoculum and substrate are added to bottles, and the headspace of each bottle is flushed to remove \ce{O2} and ensure anaerobic conditions. 
Bottles are then weighed and placed in an incubator.
Setup should follow the requirements listed in document 100 \citep{BMPdoc100req}, including the use of a positive control\footnote{Microcrystalline cellulose is required currently, but if it is unavailable, other common substrates may be useful; see \citet{kochEvaluationCommonSupermarket2020} for suggestions. But note that BMP can not be validated without microcrystalline cellulose at this time \citep{BMPdoc100req}.}
and at least 3 replicates for each substrate.

\subsection{Inoculum and substrate quantities and bottle size}
\label{sec:quantities}
Selecting quantities of inoculum and substrate, as well as bottle size (volume) is typically based on experience.
Some general guidance is given here; these suggested values can be adjusted depending on results and substrate characteristics.

The gravimetric method requires determination of small mass losses from a heavy BMP bottle, so in general, maintaining high mass loss should be a goal.
Therefore, substrate mass should generally be at least 1 g VS.
Starting with this value as a minimum, inoculum quantity can be determined based on inoculum-to-substrate ratio (ISR), which is expressed on a VS basis and should generally be 2:1.\footnote{See \citet{holligerStandardizationBiomethanePotential2016} for more discussion on this topic.}

With substrate and inoculum quantities, a bottle size can be selected.
Alternatively, given a bottle size, the largest possible quantities can be selected.
The accuracy of the gravimetric method is affected only slightly by variation in headspace pressure, and it is possible to correct for leakage of biogas. 
However, for safety (to avoid exploding bottles), for maximum precision, and to minimize possible (but perhaps unlikely) effects of high \ce{CO2} dissolution, headspace pressure should be kept below 200 kPa (2 bar) gauge pressure, or even more cautiously, 100 kPa \citep{hafnerSystematicErrorManometric2019}. 
Bottle pressures can be estimated from headspace volume and expected (or measured) biogas volume.
With some experience, the bulging of septa can be used to identify excessive headspace pressure.
The headspace volume can be estimated by assuming a density of 1 mL g$^{-1}$ for the mixture.
Additionally, to minimize the risk of foaming causing contamination of the septum, bottles are typically not filled beyond 50\%.

Using information from earlier measurements
\footnote{
\label{fn:cellrate}
  Biogas production rate depends on substrate and inoculum characteristics, and is best estimated from previous experiments.
  For microcrystalline cellulose plus inoculum, biogas production rate typically peaks around 200 mL g$^{-1}$ d$^{-1}$ (per g VS) within the first days, with values over 300 mL g$^{-1}$ d$^{-1}$ possible but rare (these values were taken from the data collected in the large IIS-BMP inter-laboratory study \citep{hafnerImprovingInterlaboratoryReproducibility2020}).
} 
and the maximum recommended headspace pressures given above, headspace volume should between 100 and 300 mL per g substrate VS.
This target can be adjusted based on expected degradation rate and biogas potential, i.e., slower, faster, or similar biogas production compared to cellulose.
But error from initial headspace\footnote{Due to a difference in density between the flushing gas and biogas.} increases with the ratio of headspace volume to total biogas production, and although a correction is available \citep{justesenDevelopmentValidationLowcost2019}, it is less accurate when residual flushing gas remains in the headspace.
Therefore headspace volumes above 300 mL g$^{-1}$ should be avoided if possible.

The ``planning'' tool in the web app OBA is helpful for quickly calculating substrate and inoculum quantities: \url{https://biotransformers.shinyapps.io/oba1/}.
This tool also checks values against the recommendations given in \citet{holligerStandardizationBiomethanePotential2016}.
Three examples for three different bottle sizes that meet the recommendations given in \citet{holligerStandardizationBiomethanePotential2016} in addition to the gravimetric recommendations given in this section are shown in Table \ref{tab:examples} below.

\begin{table}[h] 
\centering
\caption{Example quantity and volume information for three bottle sizes for gravimetric method. Note that for example C, it would not be possible to meet all recommendations if the inoculum had a VS concentration below 3.4\%.}
\label{tab:examples}
\begin{tabular}{llll}
\hline
Parameter                     & A    & B   & C   \\
\hline
Total bottle volume (mL)      & 520  & 250 & 160 \\
Inoculum VS (\% FM)           & 2.0  & 2.0 & 4.0 \\
Substrate VS (\% FM)          & 99   & 99  & 99  \\
ISR (VS basis)                & 2.0  & 2.0 & 2.0 \\
Substrate VS (g)              & 1.50 & 1.0 & 1.0 \\
Inoculum FM (g)               & 150  & 101 & 50  \\
Substrate FM (g)              & 1.52 & 1.0 & 1.0 \\
Mixture FM (g)                & 152  & 101 & 51  \\
Headspace volume (mL)         & 368  & 149 & 109 \\
Headspace:substrate VS (mL:g) & 245  & 149 & 109 \\
\hline
\end{tabular}
\end{table}

\subsection{Step-by-step instructions}
\begin{enumerate}
  \item Carefully set up and level the scale on a stable surface (following manufacturer's instructions) and check its accuracy with a weight set. 
      It is particularly important that the actual accuracy is close to reported accuracy when weighing an object with a mass close to the total mass of a BMP bottle and its contents. 
      For a scale with a reported accuracy of 50 mg, for example, this could be checked by taring the scale with a full bottle or equivalent mass, and adding a 50 mg scale calibration weight.
      Problems with accuracy or stability over time could be related to air currents, and can generally be addressed by selecting a proper location or blocking air flow with, e.g., a cardboard box.
    \item Add the required mass of inoculum and substrate (see Section \ref{sec:quantities}), along with any other additions (e.g., a trace element solution \citep{holligerStandardizationBiomethanePotential2016}) to each labeled bottle and seal with a septum and cover. 
      Determination of the quantity of material added by mass difference is the recommended approach: tare scale with bottle, add approximately the desired quantity, wipe any material from near the mouth of the bottle, and finally determine the actual quantity from the scale reading. 
      When filling the bottles, the aim is not to achieve the determined value as closely as possible, but to work in a fast (in order to minimize possible separation of the sample) and careful way (to avoid spills, errors, or safety problems) and to then record the exact mass added.
      Note that the scale used here does not need to be the same scale used for determining mass loss (see Section \ref{sec:incsam}).
    \item Flush the bottle headspace to remove \ce{O2}. 
      A simple approach is to use a needle attached to a flow meter (e.g., a rotameter), a pressure regulator (to ensure low pressure), and a gas cylinder with plastic tubing, along with a separate needle for venting. With the gravimetric method, pure \ce{N2} is preferred for flushing over \ce{N2}/\ce{CO2} mixtures.\footnote{
        Flushing gas results in an (generally small) error because its density may differ from produced biogas density (the density of \ce{N2} is identical to a \ce{CH4}:\ce{CO2} mixture with 58\% \ce{CH4} and 42\% \ce{CO2}) but this can be corrected in calculations \citep{justesenDevelopmentValidationLowcost2019}.
} 
      Minimize \ce{CO2} stripping by flushing for only 3 to 4 headspace volume exchanges. 
      Ensure that the flushing gas does not bubble through the liquid in the bottle (needle should not be submerged) to avoid \ce{CO2} stripping. 
      Allow the pressure in each bottle’s headspace to equilibrate with atmospheric pressure before removing the venting needle.
    \item Make 3 ``control'' bottles to use to to check the stability of the scale
      It is essential that they have constant mass throughout the experiment. 
      Ideally they should be a similar size and weigh about as much as the other BMP bottles.
      This can be done by adding dry sand to a BMP bottle and sealing the top with a septum. 
      Bottles filled with water have been used successfully, but in other cases have been found to lose a small amount of water.  
      Calibration weights for scales could also be used.
    \item Weigh each bottle and record as ``initial mass''. 
      Repeat this initial weighing in order to minimize the chance of a recording error, because calculations of cumulative \ce{CH4} production at all timepoints require an accurate initial mass measurement.
      If there is a discrepancy between these two initial measurements, weigh again to determine the correct mass.
      It is important that the only change in bottle mass after this time is due to biogas removal.
      Bottles should be kept clean, and labels should not be added after this time, for example.
    \item Place bottles in the incubator set at the test temperature.
\end{enumerate}

\section{Incubation and sampling}
\label{sec:incsam}
Bottles are removed from the incubator occasionally to vent and weigh in what is here referred to as a ``sampling event''.
Details on sampling frequency can be found in Section \ref{sec:freq} and step-by-step instructions for each event in Section \ref{sec:steps}.
Biogas temperature affects water vapor content. 
Although the effect is small, to minimize uncertainty in the headspace temperature used in calculations, the time that bottles spend outside the incubator should be as short as possible, and the same procedure and timing should be followed for each sampling event. 

\subsection{Sampling frequency}
\label{sec:freq}

Determining when to sample bottles in any manual BMP method, i.e., when to intermittently remove and measure accumulated biogas, is typically based on experience. 
Some general considerations are:
\begin{itemize}
  \item As long as the headspace:substrate VS ratio is sufficiently high (see Section \ref{sec:quantities}), there is no need to sample more than once per day
  \item Sampling frequency can change over time, and is generally highest at the start (1 d interval) and low at the end (maximum of perhaps a 7 d interval)
  \item As long as biogas is not lost before or during measurement due to leakage resulting from long incubation intervals, sampling frequency does not strongly affect accuracy or precision of the gravimetric method
\end{itemize}

Following these recommendations, a good approach is to sample daily from the start, and reduce the sampling frequency as biogas production slows, taking care to avoid high headspace pressure or too much accumulated biogas for the sampling system.
As mentioned above (\ref{sec:quantities}), a bulging septum can be used as an indicator of excessive headspace pressure.

\subsection{Step-by-step instructions} 
\label{sec:steps}
\begin{enumerate}
    \item Measure and record the room temperature and pressure at which biogas volume will be determined.
    \item Remove the 3 control bottles from the incubator and weigh them to confirm scale consistency. 
      If the results are the same as the initial masses (within the expected accuracy) proceed, otherwise, identify and address the problem with the scale or replace the scale if necessary.
      If the problem cannot be resolved, proceed and later correct mass results for scale drift.\footnote{
        Correction is done by subtracting the average apparent mass gain in the control bottles from all mass measurements made during that particular sampling event. 
        For example, if all three water controls weighed 0.1 g more on day 4 than at the start, the measured masses of \textit{all} bottles from day 4 should be adjusted downward by 0.1 g.
      }
    \item Remove a single set of replicates from the incubator (e.g., the three replicates for cellulose).
    \item Always starting with the same replicate (e.g., ``1'' or ``a''),\footnote{
        If this is done, the effect of gradual headspace cooling on measurement error (expected to be minor) can be confirmed by comparing BMP from individual replicates across all substrates.
      } gently swirl (not shake) the bottle for at least 10 s to mix the contents and encourage \ce{CO2} equilibration between solution and headspace. 
      During swirling, avoid contact between the liquid and the septum to prevent mass losses.\footnote{
        If the septum becomes contaminated with reacting material, a small amount may be pushed out during venting, which will result in error in the determination of mass loss.
        Generally it is easy to avoid this problem, but if it does occur, be sure to note the occurrence to help with interpretation later.
        If the loss is small and there is no noticeable difference among the replicates, the problem could be ignored. 
        Otherwise, data from this replicate should be discarded.
      }
    \item Weigh the bottle and record the result as pre-venting mass.
    \item Collect a biogas sample from the bottle using a syringe. 
      Puncture the septum with a needle attached to a syringe, and allow the syringe to fill under pressure. 
      Inject the required gas volume into a gas chromatograph for biogas composition analysis or into a gas sample container for later analysis.
    \item Vent the bottle using a needle.\footnote{
      If desired, use the manometer to ensure that the pressure of the bottle headspace after venting pressure is close to atmospheric (gauge pressure = $\pm3$ kPa).
    }    
    \item Weigh the bottle after venting, and record the result as post-venting mass. 
    \item Proceed to the next replicate (e.g., ``2'' or ``b'') and repeat steps 4 to 7.
    \item After all replicates have been mixed, weighed, vented, and weighed again, place the bottles back in the incubator.
    \item Proceed to the next set of replicates (e.g., the three replicates for substrate ``food waste A'') and repeat steps 3 to 9.
\end{enumerate}

This sequence of steps is shown in Fig. \ref{fig:steps}.\footnote{It may be helpful to print a copy this figure and post it where measurements are made.}

\begin{figure}[ht]
  \includegraphics[width=\textwidth]{figs/grav_steps.pdf}
  \caption{The data collection steps required for gravimetric measurements. Step numbers match those listed in the text, and are repeated for each sampling event.}
  \label{fig:steps}
\end{figure}

\section{Calculations}
See document 203 from the Standard BMP Methods website \citep{BMPdoc203grav} for a detailed description of calculations for the gravimetric method.
Calculations can also be carried out using the free web app OBA (\url{https://biotransformers.shinyapps.io/oba1/}) or the biogas package in R (\url{https://cran.r-project.org/package=biogas}) \citep{hafnerSoftwareBiogasResearch2018}.
Final results should always be evaluated based on current validation criteria \citep{BMPdoc100req}.

\bibliography{bib}

\end{document}
