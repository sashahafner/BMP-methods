%\documentclass[twocolumn]{article}
\documentclass[]{article}
\usepackage[version=3]{mhchem}
\usepackage{sectsty}
\usepackage{amsmath}
\usepackage[flushleft]{threeparttablex} % For table notes
\usepackage{rotating} 
\usepackage{longtable}
\usepackage{hyperref}
\usepackage{gensymb}
% For decimal alignment in tables
\usepackage{dcolumn}
\usepackage{siunitx}

% For dcolumn
\newcolumntype{.}{D{.}{.}{3}}

\sectionfont{\large}

\newcommand{\unit}[1]{\ensuremath{\, \mathrm{#1}}}

\title {Calculation of methane production from gravimetric measurements}
\author{Sasha D. Hafner
\\
\texttt{sasha.hafner@eng.au.dk} (S. D. Hafner)\\
}

\begin{document}
\maketitle

\section{BMP-methods}
File version 0.1 - \textbf{DRAFT}. 
This file is from the GitHub repository BMP-methods.
For more information, visit BMP-methods at \url{https://github.com/sashahafner/BMP-methods}.

\section{Description}
In the gravimetric BMP method, bottles are weighed after removing biogas for composition analysis \cite{validation}.
With mass loss and biogas composition, \ce{CH4} production can accurately be determined.
The standardised volume of \ce{CH4} produced and released by venting is calculated separately for each bottle and for each incubation interval, based on the observed mass loss and an estimate of biogas density and water vapour content. 
Biogas density is calculated from composition, using two possible approaches, called method 1 and method 2 here.
In method 1, vented biogas is assumed to consist of only \ce{CH4} and \ce{CO2}. 
When flushing gas density differs from biogas density, this assumption introduces systematic error (generally small), but the resulting small error can be corrected as long as virtually all of the initial headspace gas is removed from the bottle by the end of the test. 
Alternatively, method 2 assumes vented biogas is a mixture of \ce{CH4}, \ce{CO2}, and \ce{N2}, to account for common flushing gases (or mixtures, e.g., \ce{N2}, or a mix of \ce{N2} and \ce{CO2}).
Here, produced \ce{CH4} is includes two components: \ce{CH4} in vented biogas, and \ce{CH4} remaining in the bottle headspace.
This method is expected to be more accurate.
In the original publication describing the gravimetric method \cite{validation} only method 1 was presented.

\section{Calculation of CH$_4$ production}

\subsection{Method 1}

The mole fraction of CH$_{4}$ ($x_{CH_4}$, dimensionless) normalized for CH$_{4}$ and CO$_{2}$ can be calculated according Eq. (1). 

\begin{equation}
    x_{CH{4},n} = \frac{0.656}{0.656 + 0.289} = 0.694
\end{equation}

Biogas molar mass is calculated from this value.

\begin{equation}
  M_b = 16.04 x_{CH_4,n} + 44.01 (1 - x_{CH_4,n})
\end{equation}

And from molar mass and an assumption of 2300 mL mol$^{-1}$ for molar volume $V_b$ \cite{validation}, dry biogas density at 101.325 kPa, 0$^\circ$C, and dry conditions can be calculated.

\begin{equation}
  d_b=\frac{M_b}{V_b}
\end{equation}

This value of $d_b$ can be used to calculate biogas volume from mass loss, but a correction for water vapor is needed.
Water vapor pressure ($P_{H_2O}$, kPa) is assumed to be at saturation, and can be calculated using a Magnus-form equation from Alduchov and Eskridge (1996)\footnote{
  Other options exist, and most will provide nearly identical values}.

\begin{equation}
\label{eq:2_magnus}
   P_{H_2O} = 0.61094 \cdot e^{\frac{17.625 T}{243.04 + T}}
\end{equation}

\begin{equation}
  \label{eq:5}
  m_{H_2O}=M_{H_2O} \cdot \frac{P_{H_2O}}{P_{hs}-P_{H_2O}} \cdot \frac{1}{v_b}
\end{equation}

The standardized volume of vented biogas (assumed to equal the produced biogas) is given by the following equation.

\begin{equation}
  \label{eq:7}
  V_b = \frac{\Delta m_b}{d_b-m_{H_2O}}
\end{equation}

Finally, the volume of \ce{CH4} produced is given by:

\begin{equation}
  \label{eq:ch4m1}
  V_{CH_4} = x_{CH_4, n} V_b
\end{equation}

Cumulative production is taken as the cumulative sum of interval values. 

\subsection{Method 2}
The second method requires the dry standardized biogas volume within the bottle headspace after venting.
First the volume of headspace gas is converted to dry conditions at standard pressure:

\begin{equation}
  \label{eq:h2ocorrection}
  V_{dry} = V_{headspace}(P_{meas} - P_{H_2O})/101.325 \unit{kPa}
\end{equation}
where $P_{meas}$ is the measured headspace pressure (may be assumed to be ambient) and $P_{H_2O}$ the water vapor partial pressure (both in kPa).
Eq. (\ref{eq:h2ocorrection}) is an expression of Boyle's law.
The value of $P_{H_2O}$ is assumed to be the saturation vapor pressure prior to venting, and can be calculated using the equation above.

Volume is then further standardized to 273.15 K by application of Charles's law:

\begin{equation}
  \label{eq:bgstd}
  V_{std} = V_{dry} 273.15 \unit{K}/T_{meas}
\end{equation}
where $V_{std}$ is the standardized volume of gas within a bottle's headspace after venting.
With this value, \ce{CH4} within the bottle headspace can be calculated using the following equation.

\begin{equation}
  V_{CH_4} = x_{CH_4} V_{std}
\end{equation}

This is one component of \ce{CH4}.
The other is vented \ce{CH4}.

Biogas molar mass is calculated from the mole fraction of all significant biogas components.

\begin{equation}
  M_b = 16.04 x_{CH_4} + 44.01 x_{CO_2} + 28.01 x_{N_2}
\end{equation}

This estimate of $M_b$ is used as in Method 1 to calculate biogas volume.
And the volume of vented \ce{CH4} in a single interval is given by:

\begin{equation}
  \label{eq:ch4m1}
  V_{CH_4} = x_{CH_4} V_b
\end{equation}

Cumulative vented \ce{CH4} is taken as the cumulative sum of interval values. 
Total cumulative \ce{CH4} production is the sum of this venting sum and headspace \ce{CH4}.


\begin{thebibliography}{1}

\bibitem{bmpmethods}
Hafner, S.D.,
    \newblock{2019},
    \newblock{Calculation of methane production from volumetric measurements, part of the BMP-methods repository},
    \newblock{\url{https://github.com/sashahafner/BMP-methods}}

\bibitem{magnus}
Alduchov, O.A., Eskridge, R.E.,   
    \newblock{1996},
    \newblock{Improved Magnus form approximation of saturation vapor pressure.}, 
    \newblock{Journal of Applied Meteorology} 35: 601-609

\bibitem{validation}
Hafner, S.D., Rennuit, C., Triolo, J.M., Richards, B.K.,
    \newblock{2015},
    \newblock{Validation of a simple gravimetric method for measuring biogas production in laboratory experiments.},
        \newblock{Biomass and Bioenergy} 83: 297-301

\end{thebibliography}

\end{document}
